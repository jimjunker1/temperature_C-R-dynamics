% Options for packages loaded elsewhere
\PassOptionsToPackage{unicode}{hyperref}
\PassOptionsToPackage{hyphens}{url}
%
\documentclass[
]{article}
\usepackage{amsmath,amssymb}
\usepackage{lmodern}
\usepackage{setspace}
\usepackage{iftex}
\ifPDFTeX
  \usepackage[T1]{fontenc}
  \usepackage[utf8]{inputenc}
  \usepackage{textcomp} % provide euro and other symbols
\else % if luatex or xetex
  \usepackage{unicode-math}
  \defaultfontfeatures{Scale=MatchLowercase}
  \defaultfontfeatures[\rmfamily]{Ligatures=TeX,Scale=1}
\fi
% Use upquote if available, for straight quotes in verbatim environments
\IfFileExists{upquote.sty}{\usepackage{upquote}}{}
\IfFileExists{microtype.sty}{% use microtype if available
  \usepackage[]{microtype}
  \UseMicrotypeSet[protrusion]{basicmath} % disable protrusion for tt fonts
}{}
\usepackage{xcolor}
\IfFileExists{xurl.sty}{\usepackage{xurl}}{} % add URL line breaks if available
\IfFileExists{bookmark.sty}{\usepackage{bookmark}}{\usepackage{hyperref}}
\hypersetup{
  pdftitle={Warming non-randomly speeds up food web fluxes in both an absolute and relative sense},
  pdfauthor={James R. Junker1,2,; Wyatt F. Cross1; James M. Hood3; Jonathan P. Benstead4; Alex D. Huryn4; Daniel Nelson5; Jon S. Olafsson6; Gisli M. Gislason7},
  hidelinks,
  pdfcreator={LaTeX via pandoc}}
\urlstyle{same} % disable monospaced font for URLs
\usepackage[margin=1in]{geometry}
\usepackage{graphicx}
\makeatletter
\def\maxwidth{\ifdim\Gin@nat@width>\linewidth\linewidth\else\Gin@nat@width\fi}
\def\maxheight{\ifdim\Gin@nat@height>\textheight\textheight\else\Gin@nat@height\fi}
\makeatother
% Scale images if necessary, so that they will not overflow the page
% margins by default, and it is still possible to overwrite the defaults
% using explicit options in \includegraphics[width, height, ...]{}
\setkeys{Gin}{width=\maxwidth,height=\maxheight,keepaspectratio}
% Set default figure placement to htbp
\makeatletter
\def\fps@figure{htbp}
\makeatother
\setlength{\emergencystretch}{3em} % prevent overfull lines
\providecommand{\tightlist}{%
  \setlength{\itemsep}{0pt}\setlength{\parskip}{0pt}}
\setcounter{secnumdepth}{-\maxdimen} % remove section numbering
\newlength{\cslhangindent}
\setlength{\cslhangindent}{1.5em}
\newlength{\csllabelwidth}
\setlength{\csllabelwidth}{3em}
\newlength{\cslentryspacingunit} % times entry-spacing
\setlength{\cslentryspacingunit}{\parskip}
\newenvironment{CSLReferences}[2] % #1 hanging-ident, #2 entry spacing
 {% don't indent paragraphs
  \setlength{\parindent}{0pt}
  % turn on hanging indent if param 1 is 1
  \ifodd #1
  \let\oldpar\par
  \def\par{\hangindent=\cslhangindent\oldpar}
  \fi
  % set entry spacing
  \setlength{\parskip}{#2\cslentryspacingunit}
 }%
 {}
\usepackage{calc}
\newcommand{\CSLBlock}[1]{#1\hfill\break}
\newcommand{\CSLLeftMargin}[1]{\parbox[t]{\csllabelwidth}{#1}}
\newcommand{\CSLRightInline}[1]{\parbox[t]{\linewidth - \csllabelwidth}{#1}\break}
\newcommand{\CSLIndent}[1]{\hspace{\cslhangindent}#1}
\usepackage{lineno}
\usepackage{amsmath}
\usepackage{indentfirst}
\linenumbers
\ifLuaTeX
  \usepackage{selnolig}  % disable illegal ligatures
\fi

\title{Warming non-randomly speeds up food web fluxes in both an
absolute and relative sense}
\author{James R. Junker\textsuperscript{1,2,*} \and Wyatt F.
Cross\textsuperscript{1} \and James M.
Hood\textsuperscript{3} \and Jonathan P.
Benstead\textsuperscript{4} \and Alex D.
Huryn\textsuperscript{4} \and Daniel Nelson\textsuperscript{5} \and Jon
S. Olafsson\textsuperscript{6} \and Gisli M.
Gislason\textsuperscript{7}}
\date{}

\begin{document}
\maketitle

\setstretch{1}
\textsuperscript{1} Department of Ecology, Montana State University,
Bozeman, MT USA\\
\textsuperscript{2} Current \texttt{address:} Great Lakes Research
Center, Michigan Technological University, Houghton, MI USA\\
\textsuperscript{3} Department of Evolution, Ecology, and Organismal
Biology, The Ohio State University, Translational Data Analytics
Institute, The Aquatic Ecology Laboratory, Columbus, OH 43212, USA\\
\textsuperscript{4} Department of Biological Sciences, University of
Alabama, Tuscaloosa, AL 35487, USA\\
\textsuperscript{5} National Aquatic Monitoring Center, Department of
Watershed Sciences, Utah State University, Logan, UT 84322, USA\\
\textsuperscript{6} Institute of Marine and Freshwater Fisheries,
Reykjavik, Iceland\\
\textsuperscript{7} Univeristy of Iceland, Institute of Life and
Environmental Sciences, Reykjavik, Iceland

\textsuperscript{*} Correspondence:
\href{mailto:jrjunker@mtu.edu}{James R. Junker
\textless{}\href{mailto:jrjunker@mtu.edu}{\nolinkurl{jrjunker@mtu.edu}}\textgreater{}}

\newpage

\hypertarget{abstract}{%
\section{Abstract}\label{abstract}}

Warming temperatures are altering communities and trophic networks
globally. While the influence of warming in food webs is often
context-dependent, increasing temperatures are predicted to alter food
webs and ecosystems in a number of general and fundamental ways, among
them, 1) decreasing organism body size and 2) increasing metabolic
rates. Both body size and metabolic rate are important drivers of many
ecological patterns. Therefore, warming-induced changes have the
potential to propagate through countless ecosystem processes thereby
altering the distribution of food webs fluxes, food web stability, and
potentially the importance of environmental and biological factors in
community assembly. Here, we quantify the patterning and relative
distribution of organic matter fluxes through stream food webs spanning
a natural \textasciitilde25\(^\circ\)C temperature gradient. We then
related patterns in fluxes to species and community traits (body size,
\emph{M} and biomass turnover, \emph{P:B}) within and across ecosystems.
We predicted 1) fluxes would skew towards smaller \emph{M} and higher
\emph{P:B} both across and within ecosystems and 2) fluxes would be more
skewed at higher temperatures. Across the temperature gradient,
communities in warmer streams were composed of smaller-bodied and
higher-turnover populations on average. Additionally, material fluxes of
warmer streams were increasingly skewed towards high-turnover
populations \emph{within} the community, showing higher temperatures
restructured organic matter fluxes in both an absolute and relative
sense. Lastly, in warmer ecosystems the relative distribution of organic
matter fluxes appears to be increasingly `non-random', evidence of
potentially stronger selection for these traits with increasing
temperature. The emerging picture is a warmer world that is both smaller
and faster. Our study lends further support to this pattern and suggests
that temperature will become an increasingly important environmental
filter on warming communities in the future.

\newpage

\hypertarget{introduction}{%
\section{Introduction}\label{introduction}}

Increasing global temperatures are altering the provisioning and
maintenance of ecosystem services by modifying the network of
interactions among species that underpin ecosystem functions (de Ruiter
et al. 1995, Woodward et al. 2010, Brose et al. 2012, Thompson et al.
2012). Across global climate gradients, temperature mediates the
dynamics and stability of food webs (Baiser et al. 2019) by altering the
acquisition and allocation of resources among species (Zhang et al.
2017) and ultimately the magnitudes and relative distribution of biomass
and fluxes among species and trophic levels (May 1972, McCann et al.
1998, Barnes et al. 2018, Gibert 2019). Temperature's effects permeate
across levels of biological organization from its control on individual
metabolic rates (Gillooly et al. 2001, Brown et al. 2004) and biological
activity (e.g., attack rate, handling time, growth rates, etc.; Dell et
al. (2014), as an environmental niche filter of community assembly and
indirectly through the distribution of species traits, such as body
size, within and across species (Nelson et al. 2017a, Bideault et al.
2019).

In fact, reduced organismal body size with warming is considered a
`universal' response to climate change (Atkinson 1994, Daufresne et al.
2009, Gardner et al. 2011). Body size reductions arise from multiple
mechanisms acting at different levels of organization, such as the
increased relative abundance of smaller species within warmer
communities (Bergmann 1848), or smaller individuals in warmer
populations (James 1970), or through reduced body size of warmer
individuals (Atkinson 1994). Temperature-size relationships (TSR) are
likely to influence the distribution of body sizes both across and
within ecosystems, such that warmer communities should contain smaller
species and populations, on average, and smaller organisms should be
more dominant within a community. These changes can have important
implications for ecosystems pattern and processes because body size is a
strong determinant of an organism's biology (Peters 1983) influencing,
life history patterns (Altermatt 2010, Zeuss et al. 2017, Nelson et al.
2020a), and developmental (Angilletta et al. 2004) and metabolic rates
(Gillooly et al. 2001, Brown et al. 2004).

Organismal metabolic rate, in addition to being controlled by body size,
is strongly influenced by temperature (Gillooly et al. 2001). Therefore,
temperature may alter metabolic rate indirectly through reductions in
body size, as well as directly through its effects on subcellular
kinetics (Osmond et al. 2017, Bideault et al. 2019). The interactive
effect of these processes has been shown to modulate ecosystem patterns
through effects on population abundance (Bernhardt et al. 2018),
consumer-resource interactions (Bideault et al. 2019), and food web
structure (Gibert 2019). Importantly, the kinetic effects of
temperatures are tied to many biological processes (Dell et al. 2014),
among them growth rate (Gillooly et al. 2001), developmental rate (Zuo
et al. 2012, Nelson et al. 2020a), voltinism (Zeuss et al. 2017), and
population turnover rate (Brown et al. 2004, Huryn and Benke 2007). As
such, the emerging picture suggests a smaller, faster world in which
warming ``speeds up'' ecosystem processes through the compounded effects
of smaller body size and higher turnover rates.\emph{possibly simplify
the paragraph and combined with above}

Yet, considerable uncertainty remains in the predicted short- and
long-term effects of climate warming, especially at higher levels of
organization such as communities and food webs (Walther et al. 2002,
Woodward et al. 2010). For example, species interactions in food webs
may modulate species' TSRs (DeLong et al. 2014), leading to varied
responses among species and trophic levels (Ohlberger 2013).
Additionally, warming-induced body size reductions may act as a
stabilizing process in some consumer-resource interactions through
reduced consumer:resource biomass ratios (Osmond et al. 2017) and
reductions in consumer growth rates relative to resource supply
(Kozłowski et al. 2004, McCann 2011, DeLong 2012, Greyson-Gaito et al.
2022). Therefore, food web processes may impart varying selective
pressures on the distribution of body sizes and population turnover
within communities. If strong, these pressures may leave distinct
imprints on community structure and the distribution of energy fluxes
within food webs with potential implications for the stability and
functioning of ecosystems on a warming planet (Petchey et al. 1999,
Fussmann et al. 2014).

Here, we measured the patterning and distribution of energy fluxes
within invertebrate food webs across a natural stream temperature
gradient (\textasciitilde5--28\(^\circ\)C). Previous research in these
streams has shown a strong positive effect of temperature on primary
production both among streams (Demars et al. 2011, Padfield et al. 2017)
and within streams seasonally (O'Gorman et al. 2012, Hood et al. 2018).
Consumers rely largely on autochthonous resources (O'Gorman et al. 2012,
Nelson et al. 2020b) and therefore the dynamics of primary production
have a strong control on consumer energy demand (Junker et al. 2020). As
such, we predicted annual energy fluxes of consumers to scale with
among-stream patterns in resource availability and consumer energy
demand and increase with temperature across streams. We hypothesized
temperature to be a strong environmental filter of community assembly
and the distribution of energy and organic matter fluxes, especially at
the extremes. From this we made a number of predictions regarding the
patterning of energy fluxes within and across communities of different
temperature, with particular focus on two species traits: body size and
population biomass turnover (production:biomass ratios, \emph{P:B}).
First, we predict warmer temperatures should favor smaller and faster
turnover species leading to a decrease in body sizes and increase in
\emph{P:B} on average across communities. Second, within communities
energy fluxes should be dominated by smaller, higher turnover species at
warmer temperatures. Third, we hypothesized that the selective pressure
from environmental temperature on body sizes and population turnover
would be greatest at temperature extremes, therefore, the patterning of
energy fluxes will be `non-random' at coldest and warmest temperatures.

\hypertarget{methods}{%
\section{Methods}\label{methods}}

\hypertarget{invertebrate-sampling}{%
\subsection{Invertebrate sampling}\label{invertebrate-sampling}}

We sampled macroinvertebrate communities approximately monthly from July
2011 to August 2012 in four streams and from October 2010 to October
2011 in two streams used in a previous study (\emph{n} = 6 streams). The
two streams were part of an experiment beginning October 2011, therefore
overlapping years were not used to exclude the impact of experimental
manipulation (Nelson et al. 2017a, 2017b). Inter-annual comparisons of
primary and secondary production in previous studies showed minimal
differences among years in unmanipulated streams, suggesting that
combining data from different years would not significantly bias our
results (Nelson et al. 2017a, Hood et al. 2018). We collected five
Surber samples (0.023 m\textsuperscript{2}, 250 \(\mu\)m mesh) from
randomly selected locations within each stream. Within the sampler,
inorganic substrates were disturbed to \textasciitilde10 cm depth and
invertebrates and organic matter were removed from stones with a brush.
Samples were then preserved with 5\% formaldehyde until laboratory
analysis. In the laboratory, we split samples into coarse (\textgreater1
mm) and fine (\textless1 mm but \textgreater250 \(\mu\)m) fractions
using nested sieves and then removed invertebrates from each fraction
under a dissecting microscope (10--15 x magnification). For particularly
large samples, fine fractions were sub-sampled (1/2--1/16th) using a
modified Folsom plankton splitter prior to removal of invertebrates.
Subsamples were scaled to the rest of the sample assuming similar
abundance and body size distributions. Macroinvertebrates were
identified to the lowest practical taxonomic level (usually genus) with
taxonomic keys (Peterson 1977, Merritt et al. 2008, Andersen et al.
2013). Taxon-specific abundance and biomass were scaled to a per meter
basis by dividing by the Surber sampler area.

\hypertarget{secondary-production}{%
\subsection{Secondary Production}\label{secondary-production}}

Daily secondary production of invertebrate taxa was calculated using the
instantaneous growth rate method (IGR, Benke and Huryn 2017). Growth
rates were determined using taxon appropriate approaches described in
(Junker et al. 2020). Briefly, growth rates of common taxa (e.g.,
Chironomidae spp., \emph{Radix balthica}, etc.) were determined using
\emph{in situ} chambers (Huryn and Wallace 1986). Multiple individuals
(\emph{n} = 5--15) within small size categories (\textasciitilde1 mm
length range) were photographed next to a field micrometer, placed into
the stream within pre-conditioned chambers for 7--15 days, after which
they were again photographed. Individual lengths were measured from
field pictures using image analysis software (Schindelin et al. 2012),
and body lengths were converted to mass (mg ash-free dry mass
{[}AFDM{]}) using published length-mass regressions (Benke et al. 1999,
O'Gorman et al. 2012, Hannesdóttir et al. 2013). Growth rates (\emph{g},
d\textsuperscript{-1}) were calculated by the changes in mean body size
(\emph{W}) over a given time interval (\emph{t}) with the following
equation:

\[g = log_e ( W_{t+\Delta t} / W_t) / \Delta t\]

Variability in growth rates was estimated by bootstrapping through
repeated resampling of individual lengths with replacement (\emph{n} =
1000). For taxa which exhibit synchronous growth and development (e.g.,
Simuliidae spp., some Chironomidae spp., etc.), we examined temporal
changes in length-frequency distributions and calculated growth rates
and uncertainty using a bootstrap technique similar to that described in
Benke and Huryn (2017). Individual Lengths were converted to mg AFDM
using published length-mass regression cited above and size-frequency
histograms were visually inspected for directional changes in body size
through time. For each date, size-frequency distributions were resampled
with replacement and growth rates estimated from equation 1. We
prevented the calculation of negative growth rates by requiring
\(W_{t + \Delta t}\) \textgreater{} \(W_t\). To estimate growth rates of
taxa for which growth could not be estimated empirically, we developed
stream-specific growth rate models by constructing multivariate linear
regressions of empirical growth data against body size and temperature
within each stream. To estimate uncertainty in production of each taxon,
we used a bootstrapping technique that resampled measured growth rates,
in addition to abundance and size distributions from individual samples.
For each iteration, size-specific growth rates were multiplied by mean
interval biomass for each size class and the number of days between
sample dates to estimate size class-specific production. For each
interval, size classes were summed for each taxon to calculate total
population-level production. Intervals were summed to estimate annual
secondary production.

\hypertarget{diet-analysis}{%
\subsection{Diet analysis}\label{diet-analysis}}

Macroinvertebrate diets were quantified for dominant taxa in each
stream. We focused on numerically abundant taxa and/or taxa with
relatively high annual production. A minimum of five individuals were
selected from samples, and, when possible included individuals of
different size classes to account for ontogenetic shifts in diet. We
included individuals from different seasons to capture concurrent
ontogenetic and seasonal changes. For small-bodied taxa, we combined
multiple individuals (\emph{n} = 3--5) to ensure samples contained
enough material for quantification. We used methods outlined in
Rosi-Marshall (2016) to remove gut tracts and prepare gut contents for
quantification. Briefly, we removed the foregut from each individual or
collection of individuals and sonicated contents in water for 30
seconds. Gut content slurries were filtered onto gridded nitrocellulose
membrane filters (Metricel GN-6, 25 mm, 0.45 \(\mu\)m pore size; Gelman
Sciences, Ann Arbor, MI, USA), dried at 60 \(^\circ\)C for 15 min,
placed on a microscope slide, cleared with Type B immersion oil, and
covered with a cover slip. We took 5--10 random photographs under
200--400x magnification, depending on the density of particles, using a
digital camera mounted on a compound microscope. From these photographs
we identified all particles within each field and measured the relative
area of particles using image analysis software (Schindelin et al.
2012). We classified particles into six categories: diatoms, green and
filamentous algae, cyanobacteria, amorphous detritus, vascular and
non-vascular plants (e.g., bryophytes), and animal material and then
calculated the proportion of each food category in the gut by dividing
their summed area by the total area of all particles. Gut contents of
many predators were empty or contained unidentifiable, macerated prey.
For these taxa, we assumed 100\% animal material.

To estimate variability in diet compositions and to impute missing
values for non-dominant, yet present, taxa, we modeled the diet
proportions within each stream using a hierarchical multivariate model
(Fordyce et al. 2011, Coblentz et al. 2017). Here, the diet of a
consumer population, \emph{i}, in stream, \emph{j}, is a multinomial
vector,\(\overrightarrow{y_{ij}}\), of
\[\overrightarrow{y_{ij}} \sim Multinomial(\overrightarrow{p_{ij}}, n_{ij})\]
\[\overrightarrow{p_i} \sim Dirichlet(\overrightarrow{q_i} \times  \alpha) \]
where, \(\overrightarrow{p_i}\), is a vector of consumer diet
proportions, \(\overrightarrow{q_i}\) is a vector of the population's
diet proportions and \(\alpha\) is a concentration parameter of the
Dirichlet process. We used uniform priors for \(\overrightarrow{q_i}\)
and \(\alpha\),
\[\overrightarrow{q_i} \sim Dirichlet(\overrightarrow{\textbf{1}})\]
\[\alpha \sim Uniform(0,\textit{c})\] where,
\(\overrightarrow{\textbf{1}}\) is a vector of ones the same length of
basal resource types and \(\textit{c}\) is the assumed concentration
value. Models were fit in Stan with the `brms' package in R (Bürkner
2017). For non-dominant taxa, diet proportions were imputed from the
hierarchical model by resampling from posterior distributions.
Importantly, this process allowed to maintain the hierarchical structure
of the data when imputing missing values.

From modeled diet compositions, we estimated trophic redundancy within
and across stream food webs by calculating proportional similarities
(\emph{PS}; Whittaker (1952) among modeled diet estimates. Proportional
similarities were calculated as:

\[ PS = 1 - 0.5 \sum_{j=1}^S|p_{x,i} - p_{y,i}|\]

where, \emph{p\textsubscript{x,i}} is the proportion of food resource
\emph{i} in the diet of taxon \emph{x}, \emph{p\textsubscript{y,i}} is
the proportion of food resource \emph{i} in the diet of taxon \emph{y},
and there are \emph{S} food categories. Proportional similarity was
calculated across all taxa within a stream based on modeled diet
contributions from each taxon. To calculate \emph{PS} among streams we
sampled 1000 estimates of the mean stream-level diet proportions for
each stream and calculated \emph{PS} for each.

\hypertarget{organic-matter-consumption-estimates}{%
\subsection{Organic Matter Consumption
Estimates}\label{organic-matter-consumption-estimates}}

Consumption fluxes (g m\textsuperscript{-2} t\textsuperscript{-1}) were
calculated using the trophic basis of production method (TBP, Benke and
Wallace 1980). Taxon-specific secondary production estimates were
combined with diet proportions, diet-specific assimilation efficiencies,
\emph{AE\textsubscript{i}}, and assumed net production efficiencies,
\emph{NPE}, to estimate consumption of organic matter. For each food
category, \emph{i}, diet proportions were multiplied by the gross growth
efficiency (\(GGE_{i} = AE_{i} * NPE\)) to calculate the relative
production attributable to each food category. The relative production
from each food type was then multiplied by the interval-level production
and finally divided by \(GGE_{i}\) to estimate consumption of organic
matter from each food category by a consumer (Benke and Wallace 1980).
Consumption was calculated for each taxon across sampling intervals
(typically \textasciitilde1 month). Total interval consumption was
calculated by summing across all taxa, while annual consumption was
calculated by summing across all taxa and intervals. Variability in
consumption estimates was estimated through a Monte Carlo approach,
wherein bootstrapped vectors of secondary production for each taxon (see
\emph{Secondary production} methods above) were resampled and
consumption estimated with the TBP method using modeled diet proportions
(see \emph{Diet analysis} above), diet-specific assimilation
efficiencies, and net production efficiency. Variability in
\emph{AE\textsubscript{i}} was incorporated by resampling values from
beta distributions fit to median and 2.5\% and 97.5\% percentiles for
each diet item: diatoms = 0.30 (95\% percentile interval (PI):
0.24-0.36), filamentous and green algae = 0.30 (95\% PI: 0.24-0.36),
cyanobacteria = 0.10 (95\% PI: 0.08-0.12), amorphous detritus = 0.10
(95\% PI: 0.08-0.12), vascular and non-vascular plants (bryophytes) =
0.1 (95\% PI: 0.08-0.12), and animal material = 0.7 (95\% PI:
0.56-0.84)(Welch 1968, Benke and Wallace 1980, 1997, Cross et al. 2007,
2011). Variability in \emph{NPE} was incorporated by resampling values
from an assumed beta distribution with median \emph{NPE} = 0.45 (95\% PI
= 0.4-0.5). Beta distributions were fit using the `get.beta.par()'
function within the \emph{rriskDistributions} package (Belgorodski et
al. 2017).

\hypertarget{quantifying-the-distribution-of-food-web-fluxes}{%
\subsection{Quantifying the distribution of food web
fluxes}\label{quantifying-the-distribution-of-food-web-fluxes}}

\hypertarget{evenness-among-consumers}{%
\subsubsection{Evenness Among
consumers}\label{evenness-among-consumers}}

To visualize and quantify how evenly OM fluxes were distributed among
consumers within a stream, we constructed Lorenz curves (Lorenz 1905) on
ordered relative consumption fluxes, such that in a community with \(S\)
species and the relative consumption of species \emph{i}, \(p_i\), is
ordered \(p_1 \leq p_2 \leq ... p_S\). The Lorenz curve plots how a
value, in this case OM flux, accumulates with increasing cumulative
proportion of species. In a community with perfectly equal distribution
of OM consumption among species, the Lorenz curve is simply a straight
diagonal line. Deviation from perfect equality was calculated as the
Gini coefficient (Gini 1921), normalized for differences in \(S\) among
streams, \(G^*\) (Solomon 1975, Chao and Ricotta 2019):

\[ G^* = (2 \sum_{i = 1}^S ip_i -2)/(S-1)\]

where \(G^*\) represents an index of relative evenness of OM fluxes
bounded between zero and one--one representing a community with exactly
equal proportion of total community OM flux for all species (\(1/S\)),
and a value of zero where total community flux is attributed to a single
species.

\hypertarget{distribution-of-om-fluxes-in-relation-to-species-traits}{%
\subsubsection{Distribution of OM fluxes in relation to species'
traits}\label{distribution-of-om-fluxes-in-relation-to-species-traits}}

We predicted that warming would favor species with smaller body size and
higher population turnover and therefore OM fluxes would be skewed
towards small body size (\emph{M}) and higher \emph{P:B} across and
within communities. Across communities, we assessed the relationships
between mean annual temperature (\(^\circ\)C) and mean population
\emph{M} and \emph{P:B} of each community with bootstrapped linear
regressions. Here, values of population \emph{M} or \emph{P:B} were
resampled with replacement for each species within each stream. The mean
of all populations within a stream was calculated and a linear model was
fit between \emph{log\textsubscript{e}}-transformed \emph{M} or
\emph{P:B} and mean annual temperature. Response variables \emph{M} and
\emph{P:B} were transformed to meet the assumption of normally
distributed residual variation.

To quantify the distribution of energy fluxes \emph{\textbf{within}} a
community, we assessed if relative OM flux was skewed towards
populations with lower or higher relative \emph{M} or \emph{P:B}. To do
this, we ordered species based on within-stream ranking of annual
population traits (i.e, \(M\), \(P:B\)) and then calculated a measure of
skewness, \(Sk_{flux}\), based on quartiles of the distribution of OM
fluxes in relation to species traits as:

\[Sk_{flux} = f_{Q0.75} - 2f_{Q0.5} + f_{Q0.25}/ f_{Q0.75}-f_{Q0.25}\]

where, \(f_{Qx}\), is the cumulative flux at some quantile, \(Qx\), of
the community trait distribution. We repeated this analysis for all
bootstrapped estimates of OM flux in all communities. Skewness
coefficients exist in the range {[}-1, 1{]}, where -1 indicates OM
fluxes are skewed perfectly away from a trait and 1 indicates higher
relative flux is perfectly associated with higher trait values. To
determine if \(Sk_{flux}\) with population \emph{M} and \emph{P:B} was
related to mean annual stream temperature, we use bootstrapped beta
regression with a simple transformation to meet the assumptions of the
model, \((Sk_{flux} + 1)/2\), thereby standardizing values between 0 and
1. Model coefficients were back-transformed to estimate effect sizes.

To assess the strength of environmental filtering of species' traits
within a community, we quantified how likely it was to observe such a
skewed distribution in OM fluxes by random chance. Here, we predicted
that communities with strong filtering for traits would exhibit highly
skewed OM distributions that would be unlikely due to chance (i.e.,
`non-random ordering'). In contrast, weak environmental filtering would
produce OM fluxes with skew values that are likely due to random chance
(`random ordering'). To accomplish this, we first had to account for
statistical constraints that restrict the range of possible outcomes
(~i.e., feasible set, Haegeman and Loreau 2008, Diaz et al. 2021), given
the number of species and the relative distribution of OM fluxes within
a community. The number of unique orderings of species increases to
computationally intractable numbers very quickly (e.g., \(S!\), 10
species \textasciitilde3.6e\textsuperscript{6} unique orderings).
Therefore, we chose to permute a random subset of each stream
community's feasible set by randomly ordering species and calculating
the skewness in the cumulative distribution of annual OM fluxes 100,000
times in each stream. The number of random orderings was chosen as a
balance between characterizing the distribution of skewness values
within each feasible set and computational and time constrains. This
permuted set allowed us to calculate the probability of observing the
empirical skewness, \(Sk_{flux}\), in each stream compared to a random
ordering given the distribution of relative OM flux. To assess the
relationship between the probability of `non-random ordering' and
temperature, we again used beta regression on estimates of the
probability of seeing the empirical \(Sk_{flux}\) value or one more
extreme based on random ordering of species in the permuted feasible
set. Here, values can take the range between 0 and 1 so no
transformation was necessary. As necessary, when we observed many values
near 0 and 1, we implemented a zero-one inflated beta regressions in
brms (Bürkner 2017) to test if the probability of observing the
empirical \(Sk_{flux}\) was associated with mean annual temperature
across streams. We predicted increasingly `non-random' distributions in
OM fluxes with increasing temperature.

\hypertarget{results}{%
\section{Results}\label{results}}

\hypertarget{community-om-fluxes}{%
\subsection{Community OM fluxes}\label{community-om-fluxes}}

\hypertarget{evenness-of-om-fluxes-within-streams}{%
\subsection{Evenness of OM fluxes within
streams}\label{evenness-of-om-fluxes-within-streams}}

\hypertarget{om-fluxes-along-species-trait-distributions-among-sites-and-taxa}{%
\subsection{OM fluxes along species trait distributions among sites and
taxa}\label{om-fluxes-along-species-trait-distributions-among-sites-and-taxa}}

\hypertarget{discussion}{%
\section{Discussion}\label{discussion}}

\hypertarget{acknowledgements}{%
\section{Acknowledgements}\label{acknowledgements}}

We are grateful to..

\hypertarget{references}{%
\section{References}\label{references}}

\hypertarget{refs}{}
\begin{CSLReferences}{1}{0}
\leavevmode\vadjust pre{\hypertarget{ref-altermatt2010}{}}%
Altermatt, F. 2010.
\href{https://doi.org/10.1098/rspb.2009.1910}{Climatic warming increases
voltinism in {European} butterflies and moths}. Proceedings of the Royal
Society B: Biological Sciences 277:1281--1287.

\leavevmode\vadjust pre{\hypertarget{ref-andersen2013}{}}%
Andersen, T., P. S. Cranston, and J. H. Epler. 2013. Chironomidae of the
{Holarctic} region: {Keys} and diagnoses, {Part} 1. {Media Tryck},
{Lund, Sweden}.

\leavevmode\vadjust pre{\hypertarget{ref-angilletta2004}{}}%
Angilletta, M. J., Jr., T. D. Steury, and M. W. Sears. 2004.
\href{https://doi.org/10.1093/icb/44.6.498}{Temperature, {Growth Rate},
and {Body Size} in {Ectotherms}: {Fitting Pieces} of a {Life-History
Puzzle}}. Integrative and Comparative Biology 44:498--509.

\leavevmode\vadjust pre{\hypertarget{ref-atkinson1994}{}}%
Atkinson, D. 1994. Temperature and organism size: A biological law for
ectotherms? Advances in ecological research 25:1--58.

\leavevmode\vadjust pre{\hypertarget{ref-baiser2019}{}}%
Baiser, B., D. Gravel, A. R. Cirtwill, J. A. Dunne, A. K. Fahimipour, L.
J. Gilarranz, J. A. Grochow, D. Li, N. D. Martinez, A. McGrew, T.
Poisot, T. N. Romanuk, D. B. Stouffer, L. B. Trotta, F. S. Valdovinos,
R. J. Williams, S. A. Wood, and J. D. Yeakel. 2019.
\href{https://doi.org/10.1111/geb.12925}{Ecogeographical rules and the
macroecology of food webs}. Global Ecology and Biogeography
28:1204--1218.

\leavevmode\vadjust pre{\hypertarget{ref-barnes2018}{}}%
Barnes, A. D., M. Jochum, J. S. Lefcheck, N. Eisenhauer, C. Scherber, M.
I. O'Connor, P. de Ruiter, and U. Brose. 2018.
\href{https://doi.org/10.1016/j.tree.2017.12.007}{Energy {Flux}: {The
Link} between {Multitrophic Biodiversity} and {Ecosystem Functioning}}.
Trends in Ecology \& Evolution 33:186--197.

\leavevmode\vadjust pre{\hypertarget{ref-belgorodski2017}{}}%
Belgorodski, N., M. Greiner, K. Tolksdorf, and K. Schueller. 2017.
{rriskDistributions}: {Fitting Distributions} to {Given Data} or {Known
Quantiles}.

\leavevmode\vadjust pre{\hypertarget{ref-benke2017}{}}%
Benke, A. C., and A. D. Huryn. 2017. Secondary production and
quantitative food webs. Pages 235--254 Methods in {Stream Ecology}.
{Elsevier}.

\leavevmode\vadjust pre{\hypertarget{ref-benke1999}{}}%
Benke, A. C., A. D. Huryn, L. A. Smock, and J. B. Wallace. 1999.
\href{https://doi.org/10.2307/1468447}{Length-{Mass Relationships} for
{Freshwater Macroinvertebrates} in {North America} with {Particular
Reference} to the {Southeastern United States}}. Journal of the North
American Benthological Society 18:308--343.

\leavevmode\vadjust pre{\hypertarget{ref-benke1980}{}}%
Benke, A. C., and J. B. Wallace. 1980.
\href{https://doi.org/10.2307/1937161}{Trophic {Basis} of {Production
Among Net-Spinning Caddisflies} in a {Southern Appalachian Stream}}.
Ecology 61:108--118.

\leavevmode\vadjust pre{\hypertarget{ref-benke1997}{}}%
Benke, A. C., and J. B. Wallace. 1997.
\href{https://doi.org/10.1890/0012-9658(1997)078\%5B1132:TBOPAR\%5D2.0.CO;2}{Trophic
{Basis} of {Production Among Riverine Caddisflies}: {Implications} for
{Food Web Analysis}}. Ecology 78:1132--1145.

\leavevmode\vadjust pre{\hypertarget{ref-bergmann1848}{}}%
Bergmann, C. 1848. Über die {Verhältnisse} der {Wärmeökonomie} der
{Thiere} zu ihrer {Grösse}.

\leavevmode\vadjust pre{\hypertarget{ref-bernhardt2018}{}}%
Bernhardt, J. R., J. M. Sunday, and M. I. O'Connor. 2018.
\href{https://doi.org/10.1086/700114}{Metabolic {Theory} and the
{Temperature-Size Rule Explain} the {Temperature Dependence} of
{Population Carrying Capacity}}. The American Naturalist:000--000.

\leavevmode\vadjust pre{\hypertarget{ref-bideault2019}{}}%
Bideault, A., M. Loreau, and D. Gravel. 2019.
\href{https://doi.org/10.3389/fevo.2019.00045}{Temperature {Modifies
Consumer-Resource Interaction Strength Through Its Effects} on
{Biological Rates} and {Body Mass}}. Frontiers in Ecology and Evolution
7.

\leavevmode\vadjust pre{\hypertarget{ref-brose2012}{}}%
Brose, U., J. A. Dunne, Montoya José M., O. L. Petchey, F. D. Schneider,
and U. Jacob. 2012.
\href{https://doi.org/10.1098/rstb.2012.0232}{Climate change in
size-structured ecosystems}. Philosophical Transactions of the Royal
Society B: Biological Sciences 367:2903--2912.

\leavevmode\vadjust pre{\hypertarget{ref-brown2004}{}}%
Brown, J. H., J. F. Gillooly, A. P. Allen, V. M. Savage, and G. B. West.
2004. \href{https://doi.org/10.1890/03-9000}{Toward a {Metabolic Theory}
of {Ecology}}. Ecology 85:1771--1789.

\leavevmode\vadjust pre{\hypertarget{ref-burkner2017}{}}%
Bürkner, P.-C. 2017. \href{https://doi.org/10.18637/jss.v080.i01}{Brms:
{An R Package} for {Bayesian Multilevel Models Using Stan}}. Journal of
Statistical Software 80:1--28.

\leavevmode\vadjust pre{\hypertarget{ref-chao2019}{}}%
Chao, A., and C. Ricotta. 2019.
\href{https://doi.org/10.1002/ecy.2852}{Quantifying evenness and linking
it to diversity, beta diversity, and similarity}. Ecology 100:e02852.

\leavevmode\vadjust pre{\hypertarget{ref-coblentz2017}{}}%
Coblentz, K. E., A. E. Rosenblatt, and M. Novak. 2017.
\href{https://doi.org/10.1002/ecy.1802}{The application of {Bayesian}
hierarchical models to quantify individual diet specialization}. Ecology
98:1535--1547.

\leavevmode\vadjust pre{\hypertarget{ref-cross2011}{}}%
Cross, W. F., C. V. Baxter, K. C. Donner, E. J. Rosi-Marshall, T. A.
Kennedy, R. O. Hall, H. A. W. Kelly, and R. S. Rogers. 2011.
\href{https://doi.org/10.1890/10-1719.1}{Ecosystem ecology meets
adaptive management: Food web response to a controlled flood on the
{Colorado River}, {Glen Canyon}}. Ecological Applications 21:2016--2033.

\leavevmode\vadjust pre{\hypertarget{ref-cross2007}{}}%
Cross, W. F., J. B. Wallace, and A. D. Rosemond. 2007.
\href{https://doi.org/10.1890/06-1348.1}{Nutrient {Enrichment Reduces
Constraints} on {Material Flows} in a {Detritus-Based Food Web}}.
Ecology 88:2563--2575.

\leavevmode\vadjust pre{\hypertarget{ref-daufresne2009}{}}%
Daufresne, M., K. Lengfellner, and U. Sommer. 2009.
\href{https://doi.org/10.1073/pnas.0902080106}{Global warming benefits
the small in aquatic ecosystems}. Proceedings of the National Academy of
Sciences 106:12788--12793.

\leavevmode\vadjust pre{\hypertarget{ref-deruiter1995}{}}%
de Ruiter, P. C., A.-M. Neutel, and J. C. Moore. 1995.
\href{https://doi.org/10.1126/science.269.5228.1257}{Energetics,
{Patterns} of {Interaction Strengths}, and {Stability} in {Real
Ecosystems}}. Science 269:1257--1260.

\leavevmode\vadjust pre{\hypertarget{ref-dell2014}{}}%
Dell, A. I., S. Pawar, and V. M. Savage. 2014.
\href{https://doi.org/10.1111/1365-2656.12081}{Temperature dependence of
trophic interactions are driven by asymmetry of species responses and
foraging strategy}. Journal of Animal Ecology 83:70--84.

\leavevmode\vadjust pre{\hypertarget{ref-delong2012a}{}}%
DeLong, J. P. 2012. Experimental demonstration of a
{``rate\textendash size''} trade-off governing body size
optimization:10.

\leavevmode\vadjust pre{\hypertarget{ref-delong2014}{}}%
DeLong, J. P., T. C. Hanley, and D. A. Vasseur. 2014.
\href{https://doi.org/10.1111/1365-2435.12199}{Predator\textendash prey
dynamics and the plasticity of predator body size}. Functional Ecology
28:487--493.

\leavevmode\vadjust pre{\hypertarget{ref-demars2011}{}}%
Demars, B. O. L., J. R. Manson, J. S. Ólafsson, G. M. Gíslason, R.
Gudmundsdóttir, G. Woodward, J. Reiss, D. E. Pichler, J. J. Rasmussen,
and N. Friberg. 2011.
\href{https://doi.org/10.1111/j.1365-2427.2010.02554.x}{Temperature and
the metabolic balance of streams}. Freshwater Biology 56:1106--1121.

\leavevmode\vadjust pre{\hypertarget{ref-diaz2021a}{}}%
Diaz, R. M., H. Ye, and S. K. M. Ernest. 2021.
\href{https://doi.org/10.1111/ele.13820}{Empirical abundance
distributions are more uneven than expected given their statistical
baseline}. Ecology Letters n/a.

\leavevmode\vadjust pre{\hypertarget{ref-fordyce2011}{}}%
Fordyce, J. A., Z. Gompert, M. L. Forister, and C. C. Nice. 2011.
\href{https://doi.org/10.1371/journal.pone.0026785}{A {Hierarchical
Bayesian Approach} to {Ecological Count Data}: {A Flexible Tool} for
{Ecologists}}. PLOS ONE 6:e26785.

\leavevmode\vadjust pre{\hypertarget{ref-fussmann2014}{}}%
Fussmann, K. E., F. Schwarzmüller, U. Brose, A. Jousset, and B. C. Rall.
2014. \href{https://doi.org/10.1038/nclimate2134}{Ecological stability
in response to warming}. Nature Climate Change 4:206--210.

\leavevmode\vadjust pre{\hypertarget{ref-gardner2011}{}}%
Gardner, J. L., A. Peters, M. R. Kearney, L. Joseph, and R. Heinsohn.
2011. \href{https://doi.org/10.1016/j.tree.2011.03.005}{Declining body
size: A third universal response to warming?} Trends in Ecology \&
Evolution 26:285--291.

\leavevmode\vadjust pre{\hypertarget{ref-gibert2019a}{}}%
Gibert, J. P. 2019.
\href{https://doi.org/10.1038/s41598-019-41783-0}{Temperature directly
and indirectly influences food web structure}. Scientific Reports
9:5312.

\leavevmode\vadjust pre{\hypertarget{ref-gillooly2001}{}}%
Gillooly, J. F., J. H. Brown, G. B. West, V. M. Savage, and E. L.
Charnov. 2001. \href{https://doi.org/10.1126/science.1061967}{Effects of
size and temperature on metabolic rate}. Science (New York, N.Y.)
293:2248--2251.

\leavevmode\vadjust pre{\hypertarget{ref-gini1921}{}}%
Gini, C. 1921. \href{https://doi.org/10.2307/2223319}{Measurement of
{Inequality} of {Incomes}}. The Economic Journal 31:124--126.

\leavevmode\vadjust pre{\hypertarget{ref-greyson-gaito2022}{}}%
Greyson-Gaito, C. J., G. Gellner, and K. S. McCann. 2022.
\href{https://doi.org/10.1101/2022.01.27.478031}{Slower organisms
exhibit sudden population disappearances in a reddened world}. Preprint,
{Ecology}.

\leavevmode\vadjust pre{\hypertarget{ref-haegeman2008}{}}%
Haegeman, B., and M. Loreau. 2008.
\href{https://doi.org/10.1111/j.1600-0706.2008.16539.x}{Limitations of
entropy maximization in ecology}. Oikos 117:1700--1710.

\leavevmode\vadjust pre{\hypertarget{ref-hannesdottir2013}{}}%
Hannesdóttir, E. R., G. M. Gíslason, J. S. Ólafsson, Ó. P. Ólafsson, and
E. J. O'Gorman. 2013.
\href{https://doi.org/10.1016/B978-0-12-417199-2.00005-7}{Increased
{Stream Productivity} with {Warming Supports Higher Trophic Levels}}.
Advances in Ecological Research 48:285--342.

\leavevmode\vadjust pre{\hypertarget{ref-hood2018}{}}%
Hood, J. M., J. P. Benstead, W. F. Cross, A. D. Huryn, P. W. Johnson, G.
M. Gíslason, J. R. Junker, D. Nelson, J. S. Ólafsson, and C. Tran. 2018.
\href{https://doi.org/10.1111/gcb.13912}{Increased resource use
efficiency amplifies positive response of aquatic primary production to
experimental warming}. Global Change Biology 24:1069--1084.

\leavevmode\vadjust pre{\hypertarget{ref-huryn2007}{}}%
Huryn, A. D., and A. C. Benke. 2007. Relationship between biomass
turnover and body size for stream communities. Body size: the structure
and function of aquatic ecosystems. Cambridge University Press,
Cambridge, UK:55--76.

\leavevmode\vadjust pre{\hypertarget{ref-huryn1986}{}}%
Huryn, A. D., and J. B. Wallace. 1986.
\href{https://doi.org/10.4319/lo.1986.31.1.0216}{A method for obtaining
in situ growth rates of larval {Chironomidae} ({Diptera}) and its
application to studies of secondary production}. Limnology and
Oceanography 31:216--221.

\leavevmode\vadjust pre{\hypertarget{ref-james1970}{}}%
James, F. C. 1970. \href{https://doi.org/10.2307/1935374}{Geographic
{Size Variation} in {Birds} and {Its Relationship} to {Climate}}.
Ecology 51:365--390.

\leavevmode\vadjust pre{\hypertarget{ref-junker2020}{}}%
Junker, J. R., W. F. Cross, J. P. Benstead, A. D. Huryn, J. M. Hood, D.
Nelson, G. M. Gíslason, and J. S. Ólafsson. 2020.
\href{https://doi.org/10.1111/ele.13608}{Resource supply governs the
apparent temperature dependence of animal production in stream
ecosystems}. Ecology Letters 23:1809--1819.

\leavevmode\vadjust pre{\hypertarget{ref-kozlowski2004}{}}%
Kozłowski, J., M. Czarnołęski, and M. Dańko. 2004.
\href{https://doi.org/10.1093/icb/44.6.480}{Can {Optimal Resource
Allocation Models Explain Why Ectotherms Grow Larger} in {Cold}?1}.
Integrative and Comparative Biology 44:480--493.

\leavevmode\vadjust pre{\hypertarget{ref-lorenz1905}{}}%
Lorenz, M. O. 1905. \href{https://doi.org/10.2307/2276207}{Methods of
{Measuring} the {Concentration} of {Wealth}}. Publications of the
American Statistical Association 9:209.

\leavevmode\vadjust pre{\hypertarget{ref-may1972}{}}%
May, R. M. 1972. \href{https://doi.org/10.1038/238413a0}{Will a {Large
Complex System} be {Stable}?} Nature 238:413--414.

\leavevmode\vadjust pre{\hypertarget{ref-mccann2011}{}}%
McCann, K. S. 2011.
\href{https://doi.org/10.23943/princeton/9780691134178.001.0001}{Food
{Webs} ({MPB-50})}. {Princeton University Press}.

\leavevmode\vadjust pre{\hypertarget{ref-mccann1998}{}}%
McCann, K., A. Hastings, and G. R. Huxel. 1998.
\href{https://doi.org/10.1038/27427}{Weak trophic interactions and the
balance of nature}. Nature 395:794--798.

\leavevmode\vadjust pre{\hypertarget{ref-merritt2008}{}}%
Merritt, R. W., K. W. Cummins, and M. B. Berg, editors. 2008. An
{Introduction} to the {Aquatic Insects} of {North America}. Fourth.
{Kendall/Hunt Publishing Co.}, {Dubuque, IA}.

\leavevmode\vadjust pre{\hypertarget{ref-nelson2017}{}}%
Nelson, D., J. P. Benstead, A. D. Huryn, W. F. Cross, J. M. Hood, P. W.
Johnson, J. R. Junker, G. M. Gíslason, and J. S. Ólafsson. 2017a.
\href{https://doi.org/10.1002/ecy.1857}{Shifts in community size
structure drive temperature invariance of secondary production in a
stream-warming experiment}. Ecology 98:1797--1806.

\leavevmode\vadjust pre{\hypertarget{ref-nelson2017a}{}}%
Nelson, D., J. P. Benstead, A. D. Huryn, W. F. Cross, J. M. Hood, P. W.
Johnson, J. R. Junker, G. M. Gíslason, and J. S. Ólafsson. 2017b.
\href{https://doi.org/10.1111/gcb.13574}{Experimental whole-stream
warming alters community size structure}. Global Change Biology
23:2618--2628.

\leavevmode\vadjust pre{\hypertarget{ref-nelson2020}{}}%
Nelson, D., J. P. Benstead, A. D. Huryn, W. F. Cross, J. M. Hood, P. W.
Johnson, J. R. Junker, G. M. Gíslason, and J. S. Ólafsson. 2020b.
\href{https://doi.org/10.1002/ecy.2952}{Thermal niche diversity and
trophic redundancy drive neutral effects of warming on energy flux
through a stream food web}. Ecology.

\leavevmode\vadjust pre{\hypertarget{ref-nelson2020b}{}}%
Nelson, D., J. P. Benstead, A. D. Huryn, W. F. Cross, J. M. Hood, P. W.
Johnson, J. R. Junker, G. M. Gíslason, and J. S. Ólafsson. 2020a.
\href{https://doi.org/10.1111/fwb.13583}{Contrasting responses of black
fly species ({Diptera}: {Simuliidae}) to experimental whole-stream
warming}. Freshwater Biology 65:1793--1805.

\leavevmode\vadjust pre{\hypertarget{ref-ogorman2012}{}}%
O'Gorman, E. J., D. E. Pichler, G. Adams, J. P. Benstead, H. Cohen, N.
Craig, W. F. Cross, B. O. L. Demars, N. Friberg, G. M. Gíslason, R.
Gudmundsdóttir, A. Hawczak, J. M. Hood, L. N. Hudson, L. Johansson, M.
P. Johansson, J. R. Junker, A. Laurila, J. R. Manson, E. Mavromati, D.
Nelson, J. S. Ólafsson, D. M. Perkins, O. L. Petchey, M. Plebani, D. C.
Reuman, B. C. Rall, R. Stewart, M. S. A. Thompson, and G. Woodward.
2012. \href{https://doi.org/10.1016/B978-0-12-398315-2.00002-8}{Impacts
of {Warming} on the {Structure} and {Functioning} of {Aquatic
Communities}}. Pages 81--176 Advances in {Ecological Research}.
{Elsevier}.

\leavevmode\vadjust pre{\hypertarget{ref-ohlberger2013}{}}%
Ohlberger, J. 2013.
\href{https://doi.org/10.1111/1365-2435.12098}{Climate warming and
ectotherm body size \textendash{} from individual physiology to
community ecology}. Functional Ecology 27:991--1001.

\leavevmode\vadjust pre{\hypertarget{ref-osmond2017}{}}%
Osmond, M. M., M. A. Barbour, J. R. Bernhardt, M. W. Pennell, J. M.
Sunday, and M. I. O'Connor. 2017.
\href{https://doi.org/10.1086/691387}{Warming-{Induced Changes} to {Body
Size Stabilize Consumer-Resource Dynamics}}. The American Naturalist
189:718--725.

\leavevmode\vadjust pre{\hypertarget{ref-padfield2017}{}}%
Padfield, D., C. Lowe, A. Buckling, R. Ffrench-Constant, S. Jennings, F.
Shelley, J. S. Ólafsson, and G. Yvon-Durocher. 2017.
\href{https://doi.org/10.1111/ele.12820}{Metabolic compensation
constrains the temperature dependence of gross primary production}.
Ecology Letters 20:1250--1260.

\leavevmode\vadjust pre{\hypertarget{ref-petchey1999}{}}%
Petchey, O. L., P. T. McPhearson, T. M. Casey, and P. J. Morin. 1999.
\href{https://doi.org/10.1038/47023}{Environmental warming alters
food-web structure and ecosystem function}. Nature 402:69--72.

\leavevmode\vadjust pre{\hypertarget{ref-peters1983}{}}%
Peters, R. H. 1983. \href{https://doi.org/10.1017/CBO9780511608551}{The
ecological implications of body size}. {Cambridge University Press},
{Cambridge}.

\leavevmode\vadjust pre{\hypertarget{ref-peterson1977}{}}%
Peterson, B. V. 1977. Black flies of {Iceland} ({Diptera-Simuliidae}).
Canadian Entomologist 109:449--472.

\leavevmode\vadjust pre{\hypertarget{ref-rosi-marshall2016}{}}%
Rosi-Marshall, E. J., H. A. Wellard Kelly, R. O. Hall, and K. A. Vallis.
2016. \href{https://doi.org/10.1086/684648}{Methods for quantifying
aquatic macroinvertebrate diets}. Freshwater Science 35:229--236.

\leavevmode\vadjust pre{\hypertarget{ref-schindelin2012}{}}%
Schindelin, J., I. Arganda-Carreras, E. Frise, V. Kaynig, M. Longair, T.
Pietzsch, S. Preibisch, C. Rueden, S. Saalfeld, B. Schmid, J.-Y.
Tinevez, D. J. White, V. Hartenstein, K. Eliceiri, P. Tomancak, and A.
Cardona. 2012. \href{https://doi.org/10.1038/nmeth.2019}{Fiji: An
open-source platform for biological-image analysis}. Nature Methods
9:676--682.

\leavevmode\vadjust pre{\hypertarget{ref-solomon1975}{}}%
Solomon, D. L. 1975. A comparative approach to species diversity:7.

\leavevmode\vadjust pre{\hypertarget{ref-thompson2012}{}}%
Thompson, R. M., U. Brose, J. A. Dunne, R. O. Hall, S. Hladyz, R. L.
Kitching, N. D. Martinez, H. Rantala, T. N. Romanuk, D. B. Stouffer, and
J. M. Tylianakis. 2012.
\href{https://doi.org/10.1016/j.tree.2012.08.005}{Food webs: Reconciling
the structure and function of biodiversity}. Trends in Ecology \&
Evolution 27:689--697.

\leavevmode\vadjust pre{\hypertarget{ref-walther2002}{}}%
Walther, G.-R., E. Post, P. Convey, A. Menzel, C. Parmesan, T. J. C.
Beebee, J.-M. Fromentin, O. Hoegh-Guldberg, and F. Bairlein. 2002.
\href{https://doi.org/10.1038/416389a}{Ecological responses to recent
climate change}. Nature 416:389--395.

\leavevmode\vadjust pre{\hypertarget{ref-welch1968}{}}%
Welch, H. E. 1968. \href{https://doi.org/10.2307/1935541}{Relationships
between {Assimiliation Efficiencies} and {Growth Efficiencies} for
{Aquatic Consumers}}. Ecology 49:755--759.

\leavevmode\vadjust pre{\hypertarget{ref-whittaker1952}{}}%
Whittaker, R. H. 1952. \href{https://doi.org/10.2307/1948527}{A {Study}
of {Summer Foliage Insect Communities} in the {Great Smoky Mountains}}.
Ecological Monographs 22:1--44.

\leavevmode\vadjust pre{\hypertarget{ref-woodward2010}{}}%
Woodward, G., J. P. Benstead, O. S. Beveridge, J. Blanchard, T. Brey, L.
E. Brown, W. F. Cross, N. Friberg, T. C. Ings, U. Jacob, S. Jennings, M.
E. Ledger, A. M. Milner, J. M. Montoya, E. O'Gorman, J. M. Olesen, O. L.
Petchey, D. E. Pichler, D. C. Reuman, M. S. A. Thompson, F. J. F. Van
Veen, and G. Yvon-Durocher. 2010.
\href{https://doi.org/10.1016/B978-0-12-381363-3.00002-2}{Ecological
{Networks} in a {Changing Climate}}. Pages 71--138 Advances in
{Ecological Research}. {Elsevier}.

\leavevmode\vadjust pre{\hypertarget{ref-zeuss2017}{}}%
Zeuss, D., S. Brunzel, and R. Brandl. 2017.
\href{https://doi.org/10.1111/geb.12525}{Environmental drivers of
voltinism and body size in insect assemblages across {Europe}}. Global
Ecology and Biogeography 26:154--165.

\leavevmode\vadjust pre{\hypertarget{ref-zhang2017}{}}%
Zhang, L., D. Takahashi, M. Hartvig, and K. H. Andersen. 2017.
\href{https://doi.org/10.1098/rspb.2017.1772}{Food-web dynamics under
climate change}. Proceedings of the Royal Society B: Biological Sciences
284:20171772.

\leavevmode\vadjust pre{\hypertarget{ref-zuo2012}{}}%
Zuo, W., M. E. Moses, G. B. West, C. Hou, and J. H. Brown. 2012.
\href{https://doi.org/10.1098/rspb.2011.2000}{A general model for
effects of temperature on ectotherm ontogenetic growth and development}.
Proceedings of the Royal Society B: Biological Sciences 279:1840--1846.

\end{CSLReferences}

\hypertarget{appendix}{%
\section{Appendix}\label{appendix}}

Spare(d) words

\end{document}
